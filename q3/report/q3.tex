\documentclass[11pt, fleqn]{article}

\usepackage{amsmath}
\usepackage{amssymb}
\usepackage{amsthm}
\usepackage{mathtools}
\usepackage{hyperref}
\usepackage{ulem}
\usepackage{enumitem}
\usepackage[left=0.75in, right=0.75in, bottom=0.75in]{geometry}
% \usepackage{float}
\usepackage{floatrow}
\usepackage{graphicx}
\usepackage[export]{adjustbox}

\usepackage{sectsty}
\sectionfont{\centering}

\usepackage[perpage]{footmisc}

\usepackage{fancyhdr}
\pagestyle{fancy}
\fancyhf{}
\lhead{190100044 \& 190100055}
\rhead{CS 215: Assignment 5}
\renewcommand{\footrulewidth}{1.0pt}
\cfoot{Page \thepage}

\setlength{\parindent}{0em}
\renewcommand{\arraystretch}{2}%

\title{Assignment 5: CS 215}
\author{
\begin{tabular}{|c|c|}
     \hline
     Devansh Jain & Harshit Varma \\
     \hline
     190100044 & 190100055 \\
     \hline
\end{tabular}
}
\date{\today}

\begin{document}

\maketitle

\renewcommand{\arraystretch}{1}

\section*{Question 3}
\addcontentsline{toc}{section}{Question 3}
\setcounter{equation}{0}
\setcounter{figure}{0}

\textbf{Prior}: 
$$
    p(\theta) = K\left(\frac{\theta_m}{\theta}\right)^{\alpha}I(\theta \ge \theta_m)
$$
\textbf{Likelihood}:\\
Let $x_M = \max\{x_i\}_{i=1}^n$
$$
    L(\{x_i\}_{i=1}^n|\theta) = \frac{1}{\theta^n}I(\theta 
    \ge x_M)
$$
For maximizing/computing the mean, we would want Posterior to be non-zero.\\
This will be possible iff $\theta \ge \max\{x_M, \theta_m\}$\\
Let $M = \max\{x_M, \theta_m\}$\\
Thus, a constraint on $\theta$ is $\theta \ge M$\\
  
\medskip
\textbf{Evidence}:\\
$$
    \begin{aligned}
        e(\{x_i\}_{i=1}^n) &= \int_{M}^{\infty}L(\{x_i\}_{i=1}^n|\theta) p(\theta) d\theta\\
        &= \frac{K \theta_m^\alpha}{(n+\alpha-1)M^{n+\alpha-1}}
    \end{aligned}
$$

\textbf{Posterior}:
$$
    \begin{aligned}
        g(\theta|\{x_i\}_{i=1}^n) &= \frac{p(\theta)\cdot L(\{x_i\}_{i=1}^n|\theta)}{e(\{x_i\}_{i=1}^n)}\\
            &= \left(\frac{(n+\alpha-1) (M^{n+\alpha-1})}{K\theta_m^\alpha}\right) K\left(\frac{\theta_m}{\theta}\right)^{\alpha} I(\theta \ge \theta_m) \frac{1}{\theta^n} I(\theta \ge x_M)\\
            &= (n+\alpha-1)(M^{n+\alpha-1})\frac{1}{\theta^{n+\alpha}}I(\theta \ge M)\\
            &= K' \left(\frac{M}{\theta}\right)^{n + \alpha}I(\theta 
        \ge M)
    \end{aligned}
$$
This exactly matches the form of Pareto$(M, n+\alpha)$.\\
Since both the posterior and Pareto$(M, n+\alpha)$ integrate to 1, the normalizing constant is same in both, thus the posterior exactly equals Pareto$(M, n+\alpha)$.

\subsection*{ML Estimate}
Likelihood is maximized when $\theta$ takes the least value it is allowed to take, $= x_M$ (Here, we ignore the prior and thus ignore the constraints arising due to the prior)
$$
    \boxed{\theta_{ML} = x_M}
$$
\subsection*{MAP Estimate}
Mode of the Posterior Pareto($M, n+\alpha$) distribution occurs at $M$
$$
    \boxed{\theta_{MAP} = M}
$$
\subsection*{Posterior Mean Estimate}
This is equal to the mean of the Posterior Pareto($M, n+\alpha$) distribution
$$
    \boxed{\theta_{PME} = \left(\frac{\alpha + n - 1}{\alpha + n - 2}\right)M}
$$
  
\medskip
Neither the MAP estimate, nor the PME estimate tends to the ML estimate when $M \neq x_M$.\\
This is not desirable, as ideally more data should improve our estimates.\\
When $M = x_M$, then both MAP and PME estimates tends to the ML estimate as $n$ increases.

\medskip
The above result is justifiable - if we choose a prior of $\theta$ which only takes values greater than $x_M$, then we would never get $\theta$ such that it matches with ML Estimate which is $x_M$. \\
Had this not been the case and $\theta_m \le x_M$, i.e., ML estimate lies in the domain of prior function of $\theta$, then we see that both MAP and PME tend to ML estimate (MAP is equal to ML estimate in this case).

\end{document}
